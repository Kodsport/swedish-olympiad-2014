\problemname{Ståskrivbordet}
Annas mamma klagar att hon \emph{sitter} för mycket vid datorn varje
dag. Nä detta ska Anna minsann ordna. Hon har bestämt sig för att
\emph{stå} upp vid datorn istället!

Men datorskärmen står för högt upp! Anna behöver komma upp exakt
$x$ centimeter av ergonomiska skäl. Hon har $n$ plattor till sin
hjälp. Genom att lägga
ett antal av dem under vardera foten, så kan hon komma upp exakt $x$ centimeter.
Plattorna under höger fot måste alltså ha sammanlagd höjd $x$ centimeter, och plattorna under vänster fot måste
också ha sammanlagd höjd $x$ centimeter. Vad är det minsta totala antalet plattor hon behöver?

\section*{Indata}

Först kommer en rad med talen $x$ och $n$ ($10 \leq x \leq 100$, $2 \leq n \leq
20$), hur många centimeter Anna måste lägga under varje fot respektive
antalet plattor.

Därefter kommer en rad med $n$ heltal $h_i$ ($1 \leq h_i \leq 100$). Det kommer alltid att
finnas en lösning.

\section*{Utdata}
För att göra det svårare att gissa rätt svar, så ska du skriva ut två heltal $n_1$ och $n_2$,
antalet plattor hon har under vardera fot. Skriv det minsta talet först, d.v.s. se till att $n_1\leq n_2$.
Om det finns flera lösningar där totala antalet plattor är samma så kan du välja vilken som helst utav dem.

\noindent
\begin{tabular}{| l | l | p{12cm} |}
  \hline
  Grupp & Poängvärde & Gränser \\ \hline
  $1$    & $30$        & $n \leq 12$ \\ \hline 
  $2$    & $70$        & Inga ytterligare begränsingar. \\ \hline 
\end{tabular}
