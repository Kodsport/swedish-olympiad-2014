%% plainproblemname: Bokrecensioner
\problemname{Bokrecesioner}
En bokrecensent har läst $N$ böcker som ska recenseras. Varje recension ska
avslutas med att boken tilldelas ett betyg på en skala från $1$ till $M$. Det kan
vara svårt att direkt välja ett absolut betyg för varje bok, så
bokrecensenten tycker att det är mycket enklare att jämföra två böcker i
taget med varandra och beskriva vilken av dem som är bäst.

Bokrecensenten har numrerat böcker med heltal från $1$ till $N$ och vill nu
bestämma deras betyg $a_1, a_2, \dots , a_N$. För att göra det har
bokrecensenten gjort $R$ jämförelser som beskriver relationen mellan $a_i$ och
$a_j$, för några böcker $i, j$.

Bokrecensenten är nöjd med vilken betygsättning som helst, så länge alla krav
från jämförelserna är uppfyllda. Hjälp bokrecensenten att hitta en sådan
betygsättning.

\section*{Indata}

Första raden består av tre heltal, $N$ ($1 \leq N \leq 100\,000$), 
$M$ ($1 \leq M \leq 100\,000$),
$R$ ($1 \leq R \leq 500\,000$) -- antalet böcker, högsta möjliga
betyget och antalet jämförelser. Sedan följer $R$ rader med relationer som ska
vara uppfyllda. Varje sådan rad har formatet "\texttt{<i> <relation> <j>}",
som beskriver relationen mellan $a_i$ och $a_j$. $i$ och $j$ är heltal mellan
$1$ och $N$, $i \neq j$. Relationen $r$ är någon av strängarna '\texttt{<}', '\texttt{=}',
'\texttt{<=}', och detta beskriver just att $a_i$ \texttt{<relation>} $a_j$ måste gälla.
Inget par av böcker kommer att jämföras mer än en gång.

\section*{Utdata}

Skriv ut en lista med heltal $a_1, a_2, \ldots , a_N$ sådan att alla relationer håller, och
alla tal är på intervallet $[1, M]$. Om det finns flera lösningar, skriv ut vilken som helst.
Om det är omöjligt, skriv ut $-1$.

\section*{Poängsättning}
Din lösning kommer att testas på en mängd testfallsgrupper.
För att få poäng för en grupp så måste du klara alla testfall i gruppen.

\noindent
\begin{tabular}{| l | l | l |}
  \hline
  Grupp & Poängvärde & Gränser \\ \hline
  $1$   & $30$       & $R$ kan bara vara \texttt{<} \\ \hline
  $2$   & $30$       & $R$ kan bara vara \texttt{<} eller \texttt{=} \\ \hline
  $3$   & $40$       & Inga ytterligare begränsningar \\ \hline
\end{tabular}

\section*{Förklaring av exempelfall 1}
I det första indataexemplet så är \texttt{1 2 1 3 3} en giltig lösning. Detta kan verifieras
genom att se att alla tal ligger på intervallet $[1, 3]$, och att talen uppfyller de fyra relationerna $a_1 < a_2$,
$a_2 < a_4$, $a_3 < a_2$ och $a_2 < a_5$.