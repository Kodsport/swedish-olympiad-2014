%% plainproblemname: Bokrecensioner
\problemname{Bokrecesioner}

En bokrecensent har läst $N$ böcker som ska recenseras. Varje recension ska
avslutas med att boken tilldelas ett betyg på en skala från $1$ till $M$. Det kan
vara svårt att direkt att direkt välja ett absolut betyg för varje bok, så
bokrecensenten tycker att det är mycket enklare att jämföra två böcker i
taget med varandra och beskriva vilken av dem som är bäst.

Bokrecensenten har numrerat böcker med heltal från $1$ till $N$ och vill nu
bestämma deras betyg $a_1, a_2, \dots , a_N$. För att göra det har
bokrecensenten gjort $R$ jämförelser som beskriver relationen mellan $a_i$ och
$a_j$, för några böcker $i, j$.

Bokrecensenten är nöjd med vilken betygsättning som helst, så länge alla krav
från jämförelserna är uppfyllda. Hjälp bokrecensenten att hitta en sådan
betygsättning.

\section*{Indata}

Första raden består av tre heltal, $N$ ($1 \leq N \leq 100\,000$), 
$M$ ($1 \leq M \leq 100\,000$),
$R$ ($1 \leq R \leq 500\,000$) - antalet böcker, högsta möjliga
betyget och antalet jämförelser. Sedan följer $R$ rader med relationer som ska
vara
uppfyllda. Varje sådan rad har formatet "\texttt{<i> <r> <j>}",
som beskriver relationen mellan $a_i$ och $a_j$. $i$ och $j$ är heltal mellan
$1$ och $N$. $c$ är någon av strängarna '\texttt{<}', '\texttt{=}',
'\texttt{<=}'.

\section*{Utdata}

Skriv ut en lista med heltal $a_1, a_2, ... , a_N$ som uppfyller recensentens
krav. Om det finns flera lösningar, skriv ut vilken som helst. Om det är
omöjligt, skriv ut -1.

\section*{Delpoäng}

I det här problemet kan du samla en del av poängen utan att
lösa problemet fullständigt.

TODO
