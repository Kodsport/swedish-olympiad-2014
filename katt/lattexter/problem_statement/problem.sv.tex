%% plainproblemname: Låttexter
\problemname{Låttexter}

Det är välkänt att informationsinnehållet i moderna låttexter inte är särskilt
högt.\footnote{[1] http://en.wikipedia.org/wiki/The\_Complexity\_of\_Songs}
Vi kan representera en text genom en samling variabler, där varje
variabel antingen motsvarar en teckensträng eller en sammansättning av två
tidigare variabler. Den slutgiltiga texten ges då av värdet på den sista
variabeln.

PO-ledningen vill nu veta, för $Q$ olika värden på $R$, vilket det $R$:te tecknet i
låttexten är.

\section*{Indata}

På första raden står två heltal $N$ ($1 \leq N \leq 700000$) och $Q$ ($1 \leq Q
\leq 50000$).

Sedan följer $N$ rader, vardera innehållande något av följande två alternativ:

\begin{itemize}
\item En nolla och sedan ett ord: \texttt{0 <ett ord>} (högst $10$ tecken i
      ordet, enbart \texttt{a-z}) om variabeln representerar ett enkelt ord.
\itemvå heltal A och B, numren på de konkatenerade strängarna
      ($1 \leq A, B < $ nuvarande radnummer). Detta är alltså ett ord som skapas
      av två sammanslagna tidigare ord.
\end{itemize}

Därefter kommer $Q$ rader med ett heltal $R$ per rad ($1 \leq R \leq \min(10^{18},$ längd på
strängen$))$, numren på de tecken vi är intresserade av.

\section*{Utdata}

En enda rad med $Q$ tecken, de utplockade tecknen.

\section*{Delpoäng}

I det här problemet kan du samla en del av poängen utan att
lösa problemet fullständigt.

TODO
