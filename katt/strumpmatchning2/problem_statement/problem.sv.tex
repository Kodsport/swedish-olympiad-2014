%% plainproblemname: Strumpmatchning 2
\problemname{Strumpmatchning 2}

Arash har nu kommit hem från onsite-finalen i Linköping och är tillbaka i
vardagen. Nu har han precis kommit hem från ett tvättstugebesök och ska
återigen matcha strumpor. När han nu sitter där med sina $N$ strumpor
så känner han att han bara behöver $K$ par strumpor, resten kan få förbli
osorterade. Det är alltså okej om $N-2K$ strumpor förblir omatchade, tänker Arash.

Som i uppgiften \href{https://po.scrool.se/problems/strumpor}{strumpmatchning}
så har varje strumpa en färg $F_i$. Två strumpor $i$ och $j$ kan paras ihop om
skillnaden i färg strikt understiger heltalet $D$ d.v.s. $|F_{i} - F_{j}|<D$.
Men istället för att hjälpa Arash matcha så många strumpor som möjligt så ska
du hjälpa honom att hitta det minsta möjliga $D$
så att han kan matcha minst $K$ strumppar!

Arash tipsar även dig att kolla på algoritmen
\href{http://www.progolymp.se/Oldpage/ioitraning/dictionary.htm}{binärsökning}.
Han mumlar något om att du ska ``tänka på att binärsökning kan användas för mer än
att bara hitta ett element i en sorterad lista'', men vad det ska betyda får du
klura ut själv.

\section*{Indata}

Indata består av en rad med de två heltalen $N$ och $K$ ($2 \le N
\le 50\,000$, $2 \le 2K \le N$). Därefter följer en rad med $N$
heltal: $F_1, F_2, \dots, F_N$. Talen $F_i$ ligger mellan $1$
och $1\,000\,000\,000\,000\,000$ (inklusive). Tänk på att använda
tillräckligt stora heltalstyper när du arbetar med dessa!

\section*{Utdata}

Du ska skriva ut ett enda heltal: den minimala differens $D$ som gör att Arash kan
matcha minst $K$ strumppar med varandra.

\section*{Delpoäng}

I det här problemet kan du samla en del av poängen utan att lösa
problemet fullständigt.

\begin{itemize}
    \item För $40$ poäng så kommer $N$ vara maximalt $200$ och varje färg maximalt $5\,000$.
    \item För $70$ poäng så kommer $N$ vara maximalt $1000$.
\end{itemize}
