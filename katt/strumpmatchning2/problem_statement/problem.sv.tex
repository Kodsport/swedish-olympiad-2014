%% plainproblemname: Strumpmatchning 2
\problemname{Strumpmatchning 2}
Arash har nu kommit hem från onsite-finalen i Linköping och är tillbaka i
vardagen. Nu har han precis kommit hem från ett tvättstugebesök och ska
återigen matcha strumpor. När han nu sitter där med sina $N$ strumpor
så känner han att han bara behöver $K$ par strumpor, resten kan få förbli
osorterade. Det är alltså okej om $N-2K$ strumpor förblir omatchade, tänker Arash.

Varje strumpa en färg $F_i$. Två strumpor $i$ och $j$ kan paras ihop om
skillnaden i färg strikt understiger heltalet $D$ d.v.s. $|F_{i} - F_{j}|<D$.
Men istället för att hjälpa Arash matcha så många strumpor som möjligt så ska
du hjälpa honom att hitta det minsta möjliga $D$
så att han kan matcha minst $K$ strumppar!

\section*{Indata}

Indata består av en rad med de två heltalen $N$ och $K$ ($2 \le N
\le 50\,000$, $2 \le 2K \le N$).

Därefter följer en rad med $N$ heltal: $F_1, F_2, \dots, F_N$.
Talen $F_i$ ligger mellan $1$ och $10^{15}$ (inklusive).

\section*{Utdata}

Du ska skriva ut ett enda heltal: den minimala differens $D$ som gör att Arash kan
matcha minst $K$ strumppar med varandra.

\section*{Poängsättning}
Din lösning kommer att testas på en mängd testfallsgrupper.
För att få poäng för en grupp så måste du klara alla testfall i gruppen.

\noindent
\begin{tabular}{| l | l | l |}
  \hline
  Grupp & Poängvärde & Gränser \\ \hline
  $1$   & $40$       & $N \leq 200$, $F_i \leq 5\,000$ för alla $i$ \\ \hline
  $2$   & $30$       & $N \leq 1\,000$ \\ \hline
  $3$   & $30$       & Inga ytterligare begränsningar \\ \hline
\end{tabular}