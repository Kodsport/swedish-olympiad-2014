%% plainproblemname: Rövarspråket
\problemname{Rövarspråket}
I rövarspråket översätter man text man skrivit genom att ersätta alla konsonanter med <konsonant>o<konstant>, medan alla vokaler är oförändrade. T.ex. översätts \texttt{hej} till \texttt{hohejoj}, \texttt{moo} till \texttt{momoo} och \texttt{ojoj} till \texttt{ojojojoj}.

Oskar har precis lärt sig rövarspråket, men glömmer ibland bort att ändra vissa konsonanter. Du kommer att få en ursprungstext och en potentiell översättning som Oskar har gjort. Du ska avgöra om det är möjligt att Oskar översatte den första strängen till den andra enligt beskrivningen ovan.

Om Oskar hade skrivit \texttt{hejoj} kan den ursprunliga texten till exempel varit \texttt{hej} eller \texttt{hejoj}.

\section*{Indata}
Den första raden innehåller ett ord som Oskar har skrivit. Den andra raden innehåller ordet som vi undrar om det kan ha varit usprungstexten.

\section*{Utdata}
Utdata ska bestå av ett enda ord: \texttt{ja} om det är möjligt att det andra ordet är en översättning av det första, annars \texttt{nej}.

\section*{Delpoäng}
?
