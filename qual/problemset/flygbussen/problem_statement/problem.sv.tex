%% plainproblemname: Flygbussen
\problemname{Flygbussen}

Under planeringen av IOI 2014 i Taiwan insåg arrangörerna att det fanns alldeles för få bussar för att transportera alla tävlande lag från flygplatsen till hotellet - man hade bara lyckats boka en enda buss. Problemet är dock inte bussens rymlighet, utan att man inte vill låta alla lag vänta på flygplatsen för länge. Annars riskerar de tävlande att bli otåliga och irriterade, och det vill man ju förstås undvika så gott det går.

Bussen kan rymma godtyckligt många lag (den är \emph{jättestor}), och det tar exakt $K$ minuter för den att åka från flygplatsen till hotellet, och lika många minuter åt andra hållet. Det tar ingen tid alls för lagen att hoppa på eller stiga av bussen (bussen har \emph{jättestora} dörrar). Ursprungligen står bussen parkerad utanför flygplatsen.

Givet schemat för de $N$ lagens ankomster vill man därför se till så att den sammanlagda väntetiden för alla lag, det vill säga summan av antal minuter mellan ankomst och bussavgång, är så liten som möjligt. Beräkna hur liten denna väntetid kan bli om bussens avgångar planeras optimalt.

\section*{Exempelindata 1}
$N = 3, K = 2$
$A = {1, 1, 1}$

Svaret är 0 minuter. Alla tre grupper anländer samtidigt. Väntetiden är därför 0 minuter.

\section*{Exempelindata 2}
$N = 4, K = 3$
$A = {1, 1, 1, 5, 5}$

Svaret är 2 minuter. Bussen hämtar direkt upp de tre grupperna som anländer vid $t = 1$. Vid $t = 6$ är bussen tillbaka vid flygplatsen och då har det sista två lagen fått vänta 2 minuter totalt (eftersom de är två). Alternativet hade varit att låta bussen vänta fram till $t = 5$, men då hade totala väntetiden blivit 12 minuter.

\section*{Indata}
Den första raden innehåller heltalen $1 \leq N \leq 50 000$, och $1 \leq K \leq 100$.
Den andra raden innehåller $N$ heltal - tidpunkten i minuter då de olika lagen anländer.

\section*{Utdata}
Utdata ska bestå av ett enda tal, summan av väntetiden för alla lag givet att bussens avgångar planeras optimalt.

\section*{Delpoäng}
TODO: fixme
