%% plainproblemname: TV-tittande
\problemname{TV-tittande}

Johan älskar att titta på TV hela dagarna, men klarar inte av reklampauser. När det blir en reklampaus i programmet han tittar på byter han till nästa kanal, och upprepar detta ända tills han hittar en kanal utan reklam. Om han tittar på den sista kanalen börjar han istället om på den första kanalen. Sedan tittar han på denna kanal tills det blir reklam, och så vidare.

Johan undrar i efterhand hur mycket han faktiskt tittade på varje kanal under en dag. Givet mellan vilka tider det var reklampauser på varje kanal, räkna ut detta åt honom. Han tittade mellan tiderna 00:00 och ett dygn framåt, och började titta på kanal 1. Du kan anta att ett kanalbyte tar 1 minut, och att under kanalbytet sker inget tv-tittande (för varje kanalbyte så försvinner alltså en minut av tv-tittande).

\section*{Indata}
Indata börjar med ett heltal $N$, antalet kanaler. Sedan följer en sektion per kanal: varje sektion börjar med ett heltal $r_i$, antalet reklampauser på kanalen. Varje reklampaus anges på formen \texttt{HH:MM-HH:MM}, t.ex. betyder 05:24-22:04 att kanalen hade reklampaus från 05:24 fram till 22:04. Observera att under minut 22:04 så är det aktuella reklamklippet slut på kanalen.

\section*{Utdata}
Skriv ut $N$ heltal på olika rader - antalet minuter Johan tittade på varje kanal.

\section*{Delpoäng}
?
