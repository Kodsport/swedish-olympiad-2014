\problemname{Vikingahackare}
Att vikingarna var duktiga krigare känner nog de flesta till, men att de också hade bra programmerare är mindre välkänt. Att programmera en runsten krävde mycket tid och lämnade inte mycket rum för misstag. Detta gjorde tyvärr även runstenarna extra sårbara för vikingahackare, som härjade fritt.

Du har fått i uppdrag att översätta en runsten från den här tiden med hjälp av ett uppslagsverk av tecken och dess binära representation (vikingarna saknade högnivåspråk och kodade direkt i ettor och nollor). Eftersom de flesta runstenar till slut blev hackade så finns det dock risk att vissa delar av koden är fel.

\section*{Indata}
På första raden står ett tal $1 \le T \le 16$, antalet tecken i alfabetet. Därefter följer $T$ rader bestående av ett tecken (små och stora bokstäver mellan $a-z$ samt siffror) följt av tecknets binära representation (en sekvens av ettor och nollor som alltid är av längd 4). Till sist följer en rad bestående av högst $1\,000$ ettor och nollor, runstenen som ska översättas (längden av den här strängen är alltid delbar med 4). Det är garanterat att det inte finns två olika tecken med samma binära representation i indata.

\section*{Utdata}
En rad med översättningen av runstenen. För de tecken som inte kunde översättas korrekt ska istället ett \texttt{"?"} skrivas ut.

\section*{Delpoäng}
TBD
