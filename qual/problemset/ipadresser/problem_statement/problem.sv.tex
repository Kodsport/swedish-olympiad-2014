%% plainproblemname: IP-adresser
\problemname{IP-adresser}
En IPv4-address består av fyra heltal mellan 0 och 255 (som inte får ha några inledande nollor), separerade av punkter. T.ex. är 1.0.3.255 en giltig address, medan 1.0.03.255, 1.0.3.256 och 1.0.3 inte är giltiga addresser.
Givet en sekvens av siffror, hitta alla giltiga IPv4-adresser som kan skapas genom insättning av tre punkter i sekvensen.

\section*{Indata}
Indata består av en sekvens av minst 4 och max 12 siffror.

\section*{Utdata}
Den första raden ska bestå av antalet giltiga IP-adresser, $N$. De nästkommande $N$ raderna ska bestå av de giltiga IP-adresserna, sorterade i bokstavsordning (där en punkt kommer före en siffra).

\section*{Delpoäng}
I testfall värda 30 poäng består sekvens av \emph{exakt} 12 siffror.
