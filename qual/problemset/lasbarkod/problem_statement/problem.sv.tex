%% plainproblemname: Läsbar kod
\problemname{Läsbar kod}
En viktig detalj när man programmerar är att försöka göra sina
program så läsbara som möjligt. Tydlig indentering, radbrytningar på
rätt ställen och kommentarer som förklarar koden är några exempel
på knep som ofta används för att göra kod mer läsbar. Tyvärr har
inte alla samma uppfattning om vad stilren kod innebär, och man brukar
definiera ett antal stilregler som man försöker hålla sig till.

Programmeraren Mårten gillar att skriva sina program på en enda lång
rad. Han använder alltså \textbf{inga} radbrytningar någonsin.
Mårten skriver många program som löser många intressanta problem,
och det är inte ovanligt att Mårten vill skicka sitt program till dig
för att be dig hjälpa honom hitta en bugg som han har någonstans.
Det slutar alltid med att du får sitta och lägga in radbrytningar på
rätt ställen, allt för att till slut kunna förstå vad Mårtens kod
gör. Du börjar tröttna på att behöva göra detta varje gång, och
det har blivit dags att göra något åt saken.

Skriv ett program som utför följande uppgift: Givet ett program utan
radbrytningar i koden, lägg in en radbrytning efter varje semikolon,
och skriv ut det resulterande programmet.

\section*{Indata}

En sträng med max 1000 tecken på en enda rad.

\section*{Utdata}

Ett antal rader med det tillrättade programmet.

\section*{Delpoäng}

För testdata mostvarande hälften av poängen så kommer indata vara på max 100 tecken. 

TODO(arash) makear detta sense för finalen?
