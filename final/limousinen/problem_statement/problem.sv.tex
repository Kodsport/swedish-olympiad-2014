%% plainproblemname: Limousinen
\problemname{Limousinen}
I Taipei har IOI satt igång med ett intensivt schema av föreläsningar, lekar,
förberedelser, matpauser, sömnpauser och tävlingar. Men arrangörerna har lämnat
en timme fri för alla att göra vad de vill. Detta är ett problem som lagledarna
inte alls förberett de tävlande på. Förvirring och kaos uppstår i de tävlandes
hjärnor, och när fritidstimmen är slut så är de $N$ tävlande utspridda vid
olika korsningar i storstaden och har inte en aning om hur de ska hitta
tillbaka. Lyckligtvis hade tävlingsledningen installerat ett chip i varje
tävlandes väska som gör att de kan se exakt vid vilken gatukorsning som varje
vilsen tävlande befinner sig.

Limousinföraren Simon har fått i uppdrag att plocka upp de tävlande och köra
dem, en i taget, till tävlingsarenan. Han vill naturligtvis undsätta dem i en
ordning som gör att han hinner skjutsa tillbaka så många som möjligt innan
nästa föreläsning börjar (vilket sker om $T$ minuter). Taipei kan modelleras
som ett oändligt regelbundet rutnät där heltalskoordinater är korsningar och
det finns lodräta och horisontella vägar. Det tar exakt 1 minut att köra från
en korsning till en närliggande korsning.

Simon börjar vid tävlingsarenan på adressen $(0, 0)$, kör till den första
personen och hämtar upp den och kör sedan hem personen till arenan. Därefter
kör han till nästa person, o.s.v. Han fortsätter så tills tiden tar slut, och
han kan alltså bara skjutsa en i taget. Om han väljer ordningen han hämtar upp
de tävlande i optimalt, hur många hinner han hämta upp och skjutsa tillbaka
inom $T$ minuter?

\section*{Indata}
På första raden i indata står två heltal $N$ ($1 \leq N \leq 100\,000$), antalet tävlande som ska plockas upp och $T$ ($1 \leq T \leq 10^9$)
respektive hur lång tid Simon har på sig.

Sedan följer $N$ rader (en rad per tävlande). Varje rad består av
två heltal $-10^8 \leq x, y \leq 10^8$, $x$- och $y$-koordinater för personens
nuvarande position.

\section*{Utdata}
Skriv ut hur många tävlande som Simon hinner köra hem innan tiden är slut,
givet att han väljer ordningen optimalt.

\section*{Poängsättning}
Din lösning kommer att testas på en mängd testfallsgrupper.
För att få poäng för en grupp så måste du klara alla testfall i gruppen.

\noindent
\begin{tabular}{| l | l | l |}
  \hline
  Grupp & Poängvärde & Gränser \\ \hline
  $1$   & $20$       & $N \leq 10$, $-1000 \leq x,y \leq 1000$ \\ \hline
  $2$   & $30$       & $N \leq 1000$, $-10^6 \leq x,y \leq 10^6$ \\ \hline
  $3$   & $50$       & Inga ytterligare begränsningar \\ \hline
\end{tabular}

\section*{Förklaring av exempelfall}
Förklaring av exempelfall 1: Simon har 5 minuter på sig, och det finns tre tävlande att hämta upp. Att hämta den första tävlande tar 4 minuter, att hämta den andra tar 6 minuter, och att hämta den trejde tar 4 minuter (om han åker optimalt). Eftersom han bara har 5 minuter på sig så hinner han då bara hämta en person.

Förklaring av exempelfall 2: Simon hinner inte hämta någon alls.

Förklaring av exempelfall 3: Simon hinner precis hämta en person, personen som befinner sig vid $(-100, 0)$.