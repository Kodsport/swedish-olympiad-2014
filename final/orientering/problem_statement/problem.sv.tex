%% plainproblemname: Orientering
\problemname{Orientering}

Springoalla har börjat med orientering, men är ärligt talat inte
särskilt bra på det. Faktum är att hon trots de vägvisande pilarna
som satts upp springer vilse nästan varenda gång.

Skogen hon springer i kan ses som ett rektangulärt $M$x$N$ rutnät, med
pilar av fyra olika sorter utsatta: \texttt{\^}, \texttt{>}, \texttt{v},
\texttt{<}. Tecken \texttt{.} används för rutor som är markerade med pilar.

Springoalla kommer in på den övre vänstra rutan, springandes åt
höger. När hon kommer till en pil byter hon automatiskt riktning och
börjar springa åt det håll pilen pekar. Det händer dock ibland att
hon missar en pil, och i stället fortsätter rakt förbi den.

Givet en position i skogen, hur många pilar måste Springoalla minst ha missat för att hamna där? Notera att hon aldrig kan ha sprungit ut ur skogen och att huruvida hon missar en pil inte påverkas av om hon varit på platsen tidigare (om hon t.ex. missar pilen två gånger räknas det som två missar).

\section*{Indata}

På första raden står fyra tal $N$, $M$, $R$ och $C$, höjden och bredden på
skogen, samt positionen (rad och kolumn) Springoalla slutar på (raderna
är numrerade från 1 till $N$ och kolumnerna från 1 till $M$). Därefter
följer $N$ rader med $M$ tecken, som beskriver skogens utseende. Varje
tecken kommer att vara antingen \texttt{.}, \texttt{v}, \texttt{\^}, \texttt{<} eller \texttt{>}.

\begin{itemize}
\item \texttt{\^} representerar upp
\item \texttt{>} representerar höger
\item \texttt{v} representerar ner
\item \texttt{<} representerar väster
\end{itemize}

Höjden och bredden kommer vara max 2000. Raden och kolumnen som
Springoalla slutar på kommer vara inom skogens koordinater.

\section*{Utdata}

Ett enda tal: Det minsta antalet pilar Springoalla måste ha missat.

\section*{Exempel}

För det första exempelindatat så har hon missat alla nedåtpilar utom en.  För
det andra exempelindatat så har hon missat mittenpilen två gånger

\section*{Delpoäng}

I det här problemet kan du samla en del av poängen utan att
lösa problemet fullständigt.

\begin{itemize}
    \item För 50 poäng så kommer $N$ och $M$ vara maximalt 200.
\end{itemize}
