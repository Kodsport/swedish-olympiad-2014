%% plainproblemname: Översättning
\problemname{Översättning}
Ett enkelt översättningsprogram används för att översätta texter mellan två
olika språk, kalla dem $A$ och $B$. Programmet översätter texten ord för ord enligt
en ordlista. När ett ord ska översättas letar programmet upp den första
förekomsten av ordet i ordlistan och översätter enligt den. 

Om vi med ett sådant program översätter en text från $A$ till $B$ och sedan tillbaka till $A$ är det inte säkert att vi får tillbaka precis samma text. I den här uppgiften ska du upprepa denna procedur tills texten inte ändrar sig längre. Närmare bestämt:
\begin{enumerate}
    \item Låt $a$ vara en text (en samling ord) i språket $A$.
    \item Översätt $a$ från $A$ till $B$, kalla den nya texten $b$.
    \item Översätt $b$ från $B$ till $A$, kalla den nya texten $a'$.
    \item Om $a \not = a'$, börja om från steg 1 med $a'$. Annars är texten färdigöversatt.
\end{enumerate}

Givet en ordlista och en text, skriv ut texten som skapas av ovanstående algoritm.

\section*{Indata}
På första raden står ett heltal n ($1 \leq N \leq 50\,000$), antalet ord i ordlistan.
Sedan följer $N$ rader med två ord (bestående av endast tecken \texttt{a-z},
inga mellanslag) på varje rad som beskriver hur ord ska översättas. Ett ord är maximalt 20 tecken långt.
Det första ordet är i språket $A$ och det andra i språket $B$.  Sedan följer ett heltal $1 \leq M \leq 100\,000$,
antalet ord i textsträngen som ska översättas. Sista raden innehåller textsträngen som ska översättas
bestående av ord separerade av mellanslag. Det är garanterat
att alla ord i texten finns i ordlistan.

\section*{Utdata}
Skriv ut en enda rad, texten som ges av algoritmen när den är färdig.

\section*{Exempel}
För det första exempeltestfallet så kommer algoritmen översätta meningen i följande steg:

\begin{enumerate}
    \item \texttt{programmering ar valdigt roligt}
    \item \texttt{programmering ar valdigt skoj}
    \item \texttt{programmering ar valdigt kul}
\end{enumerate}

\noindent
\begin{tabular}{| l | l | l |}
  \hline
  Grupp & Poängvärde & Gränser \\ \hline
  $1$   & $40$       & $N \leq 100$, $M \leq 100$ \\ \hline
  $1$   & $30$       & $N \leq 1000$ \\ \hline
  $3$   & $30$       & Inga ytterligare begränsningar \\ \hline
\end{tabular}
