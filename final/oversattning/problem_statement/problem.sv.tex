%% plainproblemname: Översättning
\problemname{Översättning}
Ett enkelt översättningsprogram används för att översätta texter mellan två
olika språk, kalla dem A och B. Programmet översätter texten ord för ord enligt
en ordlista. När ett ord ska översättas letar programmet upp den första
förekomsten av ordet i ordlistan och översätter enligt den. När man översätter
på det sättet är det vanligt att textens innehåll förändras något. För att se
om det är fallet kan man översätta texten upprepade gånger. Upprepningsprogrammet
följer denna algoritm:

\begin{enumerate}
    \item Låt $a$ vara en text (en samling ord) i språket $A$.
    \item Översätt $a$ från $A$ till $B$, kalla den nya texten $b$.
    \item Översätt $b$ från $B$ till $A$, kalla den nya texten $a'$.
    \item Om $a \not = a'$, börja om från steg 1 med $a'$. Annars är texten färdigöversatt.
\end{enumerate}

Givet en ordlista och en text, skriv ut texten som skapas av ovanstående algoritm.

\section*{Indata}
På första raden står ett heltal n ($1 \leq N \leq 1\,000$), antalet ord i ordlistan.
Sedan följer $N$ rader med två ord (bestående av endast tecken \texttt{a-z},
inga mellanslag) på varje rad som beskriver hur ord ska översättas. Det första
ordet är i språket $A$ och det andra i språket $B$. Sista raden innehåller en textsträng
bestående av ord separerade av mellanslag, texten som ska översättas. Det är garanterat
att alla ord i texten finns i ordlistan.

\section*{Utdata}
Skriv ut en enda rad, texten som ges av algoritmen när den är färdig.

\section*{Exempel}
TODO

\section*{Delpoäng}
I det här problemet kan du samla en del av poängen utan att lösa problemet fullständigt.
TODO
